\documentclass[a4paper,10pt]{book}
\usepackage[utf8]{inputenc}
\usepackage{graphicx}
\usepackage{amsmath}

%opening
\title{}
\author{}

\begin{document}
\chapter{18 de setembro}

\section{Definição}

A transformada de Laplace é definida como:
$$F(s)=\mathcal{L}\{ f(t)\} := \int_0^\infty f(t)e^{-st}dt,~~\Re(s)>s_0$$

 Exemplo:
 
 $$f(t)=1$$
 \begin{eqnarray*}
  F(s) &=& \int_0^\infty f(t)e^{-st}dt\\
  &=&\int_0^\infty e^{-st}dt\\
  &=&\left.\frac{e^{-st}}{-s}\right|_{t=0}^{t \to \infty}\\
  &=& \frac{0-1}{-s}=\frac{1}{s}, ~~s>0.
 \end{eqnarray*}
% 

Exemplo:
 $$f(t)=e^{as}$$
 
 
 \begin{eqnarray*}
  F(s) &=& \int_0^\infty f(t)e^{-st}dt\\
  &=&\int_0^\infty e^{at}e^{-st}dt\\
    &=&\int_0^\infty e^{(a-s)t}dt\\
    &=&\left.\frac{e^{(a-s)t}}{a-s}\right|_0^\infty\\
    &=&\frac{0-1}{a-s}\\
    &=&\frac{1}{s-a},~~~s>a.
 \end{eqnarray*}
% 
% 
% 
 \section{Propriedade da linearidade}
% 
 $$\mathcal{L}\{\alpha f(t)+\beta g(t)\}=\alpha\mathcal{L}\{ f(t)\}+\beta\mathcal{L}\{ g(t)\}$$

 exemplo:
 
 \begin{eqnarray*}
\mathcal{L}\{3+2 e^{t}\}&=&3\mathcal{L}\{ 1\}+2\mathcal{L}\{ e^t\} \\
&=&\frac{3}{s}+\frac{2}{s-1}
 \end{eqnarray*}

 % 
% 
 \section{Propriedade da derivada}
 $$\mathcal{L}\{f'(t)\}=s\mathcal{L}\{ f(t)\}-f(0)$$

 Exemplo:
 
 $$f(t)=t$$
 $$f'(t)=1$$
 
 $$\mathcal{L}\{1\}=s\mathcal{L}\{ t\}-0$$
 $$\mathcal{L}\{ t\} = \frac{1}{s^2}$$


  Exemplo:
 
 $$f(t)=t^2$$
 $$f'(t)=2t$$
 
 $$\mathcal{L}\{2t\}=s\mathcal{L}\{ t^2\}-0$$
 $$\mathcal{L}\{ t^2\} = \frac{2}{s^3}$$

 
 Analogamente:
 $$\mathcal{L}\{ t^3\} = \frac{6}{s^4}$$

  $$\mathcal{L}\{ t^n\} = \frac{n!}{s^{n+1}}$$
  
  
  
  Aplicação:
  
  $$f'(t)+f(t) = 1$$
  com $f(0)=1$.

  Aplicando a transformada de Laplace, temos:
  
  $$\left[sF(s)-f(0)\right]+F(s)=\frac{1}{s}$$
  
  $$F(s)(s+1) = \frac{1}{s}+1$$
  
  $$F(s) = \frac{1}{s(s+1)}+\frac{1}{(s+1)}$$
  
  $$F(s) = \frac{1+s}{s(s+1)}=\frac{1}{s}$$
  
  $$f(t)=1$$
  
  OBS: $f(t)=1, t\neq 2$,  $f(2)=0$.
  
  
  \chapter{21 de setembro}

  \section{A transformada inversa}
  
  Se $\mathcal{L}\{f(t)\}=F(s)$, dizemos que $f(t)$ é a transformada inversa de $F(s)$:
  $$\mathcal{L}^ {-1}(F(s))=f(t)$$
  
  \section{Propriedade da derivada - derivada segunda}
 Vimos a propriedade da derivada:
    $$\mathcal{L}\{f'(t)\}=s\mathcal{L}\{ f(t)\}-f(0)$$
 
 Agora aplicamos à derivada da função $f'(t)$:
 \begin{eqnarray*}
 \mathcal{L}\{f''(t)\}&=&s\mathcal{L}\{ f'(t)\}-f'(0)\\
 &=&s\left[sF(s)-f(0)\right]-f'(0)\\
 &=&s^2F(s)-sf(0)-f'(0)\\
 \end{eqnarray*}

 Analogamente:
 \begin{eqnarray*}
 \mathcal{L}\{f'''(t)\}
  &=&s^3F(s)-s^2f(0)-sf'(0)-f''(0)\\
 \end{eqnarray*}

 
 % 
 Exemplo:
 \begin{eqnarray*}f(t)&=&\cos(\omega t)\\
 f'(t)&=&-\omega\sin(\omega t)\\
 f''(t)&=&-\omega^2\cos(\omega t)
 \end{eqnarray*}
 
 isto é:
 $$f''(t)=-\omega^2f(t)$$
 Aplicando a transformada de Laplace, temos:
 \begin{eqnarray*}
 \mathcal{L}\{f''(t)\}&=&-\omega^2\mathcal{L}\{ f(t)\}.
 \end{eqnarray*}
 Usamos a propriedade da derivada (segunda):
 \begin{eqnarray*}
 s^2F(s)-sf(0)-f'(0)&=&-\omega^2F(s)
 \end{eqnarray*}
 isto é:
 \begin{eqnarray*}
 (s^2+\omega^2)F(s)=sf(0)+f'(0)=s
 \end{eqnarray*}
 Portanto:
 $$F(s)=\mathcal{L}\{\cos(\omega t)\}=\frac{s}{s^2+\omega^2},~~s>0$$
% 
 Analogamente, temos:
 $$\mathcal{L}\{\sin(\omega t)\}=\frac{w}{s^2+\omega^2}$$
 
 \section{Método das frações parciais para calcular transformadas inversas
 }
% * Ler seção 3.4 do livro. %https://www.ufrgs.br/reamat/TransformadasIntegrais/livro-tl/apdleatdd-mx00e9todo_das_frax00e7x00f5es_parciais_para_calcular_transformadasinversas.html
% 
 \begin{eqnarray*}F(s)&=&\frac{s^2-6s+4}{s^3-3s^2+2s}\\
 &=&\frac{s^2-6s+4}{s(s^2-3s+2)}\\
 &=&\frac{s^2-6s+4}{s(s-1)(s-2)}\\
 &=&\frac{A}{s}+\frac{B}{s-1}+\frac{C}{s-2}
 \end{eqnarray*}
% 
 O teorema das frações parciais garante que existem constantes $A$, $B$ e $C$ tais que:
 \begin{eqnarray}\label{eq_or_fp}\frac{s^2-6s+4}{s(s-1)(s-2)}
 &=&\frac{A}{s}+\frac{B}{s-1}+\frac{C}{s-2}
 \end{eqnarray}
 para todo $s$ complexo.

 
Primeiro multiplicamos (\ref{eq_or_fp}) por $s$:
\begin{eqnarray*}\frac{s^2-6s+4}{(s-1)(s-2)}
 &=&A+\frac{Bs}{s-1}+\frac{Cs}{s-2}
 \end{eqnarray*}
Substituindo $s$ por $0$, temos:
\begin{eqnarray*}\frac{4}{(-1)(-2)}
 =A~~~\Longrightarrow ~~ A=2
 \end{eqnarray*}

 
Agora multiplicamos a expressão (\ref{eq_or_fp}) por $s-1$:
% 
 \begin{eqnarray*}\frac{s^2-6s+4}{s(s-2)}
 &=&\frac{A(s-1)}{s}+{B}+\frac{C(s-1)}{s-2}
 \end{eqnarray*}

 Substituindo $s$ por 1, temos:

  \begin{eqnarray*}\frac{1-6+4}{1(1-2)}
 =B~~~\Longrightarrow ~~B = 1
 \end{eqnarray*}

 Finalmente multiplicamos (\ref{eq_or_fp}) por $s-2$:
 % 
 
 \begin{eqnarray*}\frac{s^2-6s+4}{s(s-1)}
 &=&\frac{A(s-2)}{s}+\frac{B(s-2)}{s-1}+{C}
 \end{eqnarray*}
E substuimos por $s=2$:

 \begin{eqnarray*}\frac{4-12+4}{2(2-1)}
 ={C}~~~\Longrightarrow ~~C = -2
 \end{eqnarray*}

 
 \begin{eqnarray}F(s)&=&\frac{s^2-6s+4}{s(s-1)(s-2)}\\
 &=&\frac{2}{s}+\frac{1}{s-1}-\frac{2}{s-2}
 \end{eqnarray}
 
 Olhanda na tabela, encontramos:
 $$f(t)=2+e^t-2e^{2t},~~t\geq 0$$
 Tabela com item 1 e item 7 com $a=1$ e $a=2$.
 
 {\bf Obs:}
 
 $$F(s)=\frac{s}{(s^2+1)(s-2)^3}=\frac{A+Bs}{s^2+1}+\frac{C}{(s-2)}+\frac{D}{(s-2)^2}+\frac{E}{(s-2)^3}$$
 % 
  \section{Propriedade de translação no eixo s}
% 
  Se $F(s)$ é a transformada de Laplace de $f(t)$ definida para $s>s_0$, então $e^{at}f(t)$ é a transformada inversa de $F(s-a)$, isto é
  \begin{equation*}
  \mathcal{L}\left\{e^{at}f(t)\right\} =F(s-a),\qquad s>s_0+a
  \end{equation*}
  A demostração vem da aplicação da definição da transformada de Laplace $F(s-a)$:
 \begin{eqnarray*}
   F(s-a)&=&\int_0^\infty f(t)e^{-(s-a)t}dt\\
  &=&\int_0^\infty f(t)e^{at}e^{-st}dt\\
&=&\int_0^\infty \left(f(t)e^{at}\right)e^{-st}dt\\
    &=&\mathcal{L}\left\{e^{at}f(t)\right\}
   \end{eqnarray*}

{\bf Exemplo:}
$$\mathcal{L}\left\{t^2\right\}=\frac{2}{s^3}$$
   
$$\mathcal{L}\left\{t^2e^{at}\right\}=\frac{2}{(s-a)^3}$$
   
   %  
  \section{Oscilador harmônico}
%  %https://www.ufrgs.br/reamat/TransformadasIntegrais/livro-tl/apdtedtd-aplicax00e7x00e3o_oscilador_harmx00f4nico.html
$$F(s)=\frac{1}{ms^2+\gamma s+ \kappa}$$


Caso $m=1$, $\gamma=0$, $\kappa =4$:

$$F(s)=\frac{1}{s^2+ 4}=\frac{1}{s^2+ 2^2}$$

$$f(t)=\frac{1}{2}\sin(2t)$$

Caso $m=1$, $\gamma=2$, $\kappa =5$:

\begin{eqnarray*}F(s)&=&\frac{1}{s^2+2 s+ 5}\\
&=&\frac{1}{\underbrace{(s+1)^2}_{s^2+2s+1}+4}\\
&=&\frac{1}{(s+1)^2+2^2}\\
&=&G(s+1)
 \end{eqnarray*}
onde $G(s)=\frac{1}{s^2+2^2}$
 
 Como $$g(t)=\mathcal{L}^{-1}\{G(s)\}=\frac{1}{2}\sin(2t)$$ $$f(t)=\frac{1}{2}e^{-t}\sin(2t)$$
Onde usamos o produto notável: $$(s+a)^2=s^2+2as+a^2.$$
 
 Caso $m=1$, $\gamma=3$, $\kappa=2$:
\begin{eqnarray*}F(s)&=&\frac{1}{s^2+3 s+ 2}\\
F(s)&=&\frac{1}{(s+1)(s+2)}\\
 \end{eqnarray*}

 Usando a tabela, encontramos:
 $$f(t)=e^{-t}-e^{-2t}$$
 
 Pergunta: Quantas vezes $f(t)$ para por zero para $t\geq 0$.
 $$f(t)=e^{-t}-e^{-2t}=0$$
 $$e^{-t}(1-e^{-t})=0$$
  Assim $f(t)=0$ se e somente se $e^{-t}=1$, i.e., $t=0$.
  \end{document}
  
